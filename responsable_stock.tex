\newpage
\section{Objectif : Responsable des stocks}
Le responsable des stocks cherche à minimiser le nombre de de produits dans
son stock sous les contraintes définies précedemment.

\subsection{Modélisation}
Soit $Stock(n_{i})$ le nombre d'unités de stock nécessaires pour stocker les
produits fabriqués et la matière première nécessaire. Cette fonction est la
somme des produits fabriqués à laquelle on ajoute la quantité de matières
premières nécessaire à la fabrication.

On suppose qu'un produit frabriqué correspond à une unité de stock.

On a ainsi la formule suivante, où $n_{i}$ est la quantité de produit usiné
(pour chaque produit $i$), et $Q_{j,i}$ est la quantité de matière première
par produit pour chaque produit $i$ et chaque matière première $j$.

\begin{equation}
	Stock(n) = \sum_{i} (n_{i} + n_{i} \times \sum_{j} Q_{j,i})
\end{equation}

La représentation matricielle de cette fonction sera alors :

\begin{equation}
	M_S = \begin{pmatrix}
		1 & 1 & 1 & 1 & 1 & 1
	\end{pmatrix} + (
	\begin{pmatrix}
		1 & 1 & 1
	\end{pmatrix}
	\times Q)
\end{equation}

\subsection{Décisions}
$\Delta Stock(n_{i})$ est minimum quand le maximum d'unités de matières
premières est écoulé. Ainsi, la combinaison optimale favorise la production des
produits consommant le plus de matières premières de manière à en éliminer le
plus (il vaut mieux par exemple fabriquer deux produits consommant 6 unités de
matières premières qu'un seul en consommant 4).