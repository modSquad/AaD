\documentclass[a4paper,11pt]{article} 
\usepackage[english, french]{babel}
\usepackage[utf8]{inputenc}

\usepackage{graphicx}
\usepackage{fancyhdr}
\usepackage{lastpage}
\usepackage{amsmath}
\usepackage{xspace}
\usepackage{textcomp}

\usepackage{hyperref}

\usepackage[top=20mm, bottom=20mm, left=25mm, right=25mm]{geometry}

\pagestyle{fancy}

\usepackage{helvet}

\usepackage{verbatim}
\usepackage{amsmath}
\usepackage[table]{xcolor}
\definecolor{bleugris}{rgb}{.2,.4,.5}

\definecolor{colKeys}{rgb}{0,0,1} 
\definecolor{colIdentifier}{rgb}{0,0,0} 
\definecolor{colComments}{rgb}{0,0.5,1} 
\definecolor{colString}{rgb}{0.6,0.1,0.1} 

\usepackage{listings}
\lstset{%configuration de listings 
float=hbp,% 
basicstyle=\ttfamily\small, % 
identifierstyle=\color{colIdentifier}, % 
keywordstyle=\color{colKeys}, % 
stringstyle=\color{colString}, % 
commentstyle=\color{colComments}, % 
columns=flexible, % 
tabsize=2, % 
frame=trBL, % 
frameround=tttt, % 
extendedchars=true, % 
showspaces=false, % 
showstringspaces=false, % 
numbers=left, % 
numberstyle=\tiny, % 
breaklines=true, % 
breakautoindent=true, % 
captionpos=b,% 
xrightmargin=-1cm, % 
xleftmargin=-1cm 
} 
\lstset{language=octave} 
\lstset{commentstyle=\textit}

\newcommand{\nTitle}[1]{%
	\newpage 			%
	\vspace*{\fill}		%
	\begin{center}	%
		\part{#1}		%
	\end{center}
	\vspace*{\fill}		%
	\newpage			%
}

\newenvironment{nAbstract} 		%
{ 								%
	\newpage 					% 
	\vspace*{\fill}				%
	\begin{center}			 	%
		\begin{abstract}		%
}{								%
		\end{abstract}			%
	\end{center}				%
	\vspace*{\fill}				%
	\newpage					%
}


\newcommand{\nClass}[1]{{\color{bleugris}{\textsl{\textbf{#1}}}}}
\newcommand{\nParameter}[1]{{\color{gray}{\textbf{#1}}}}
\newcommand{\nMethod}[1]{{\color{gray}{\textbf{#1}}}}
\newcommand{\nConstant}[1]{\texttt{\uppercase{#1}}}
\newcommand{\nKeyword}[1]{\textsl{\textbf{#1}}}

 
%opening
\title{\textbf{Stratégie de production}\\Optim}
\author{Paul \textsc{Adenot}, Etienne \textsc{Brodu}, Maxime \textsc{Gaudin},\\
Monica \textsc{Golumbeanu}, Nor \textsc{El Malki}, Yoann \textsc{Rodière}}
\lhead{Hexanome 4203}
\cfoot{\thepage\ de \pageref{LastPage}}

\begin{document}
\maketitle
\newpage

\vspace*{\fill}
\tableofcontents 
\vspace*{\fill}
\newpage

\begin{nAbstract}
L'objet de ce rapport est de présenter une solution capable de trouver une
stratégie de production, menant à un contentement optimal des différent
acteurs :
\begin{itemize}
  \item Le comptable
  \item Le responsable des ateliers
  \item Le responsable des stocks
  \item Le responsable commercial
\end{itemize}
~\\
La démarche sera quant à elle incrémenta le puisque nous fournirons dans un
premier temps des optimum très locaux, respectant un nombre limités de
contraintes et de critères, puis nous combinerons ces différents résultat afin
de converger vers une solution prenant en compte l'intérêt de chacun.
\end{nAbstract}

\nTitle{Programmation Linéaire monocritère}

\section{Données}
Soient :
\begin{itemize}
  \item \textbf{T} la matrice des temps unitaires d'usinage d'un produit sur une
  machine (minutes) (\textsl{C.f. Table 1}).
  \item \textbf{Q} la matrice de quantité de matières premières par produit
  (\textsl{C.f. Table 2}).
  \item \textbf{S} la matrice des quantité maximum de matières premières
  (\textsl{C.f. Table 3}).
  \item \textbf{V} la matrice des prix de vente des produits finis (\textsl{C.f.
  Table 4})
  \item \textbf{A} la matrice des prix d'achat des matières premières.
  \item \textbf{C} la matrice des coûts horaires des machines (\textsl{C.f.
  Table 5}).
\end{itemize}

\subsection{Contraintes}
Considérons :
\begin{itemize}
  \item 7 machines $j \in {1, 2, 3, 4, 5, 6 ,7}$
  \item 6 produits $i \in {A, B, C, D, E, F}$
  \item $n_i$ le nombre de d'unités $i$ fabriquées
\end{itemize}
~\\
L'ensemble de la chaine de production est régie par les contraintes suivantes
:\\
\begin{itemize}
  \item \textbf{Le nombre de produits usinés :} Il doit être non nul
  \begin{equation} 
  	\forall i, n_i \ge 0 \label{C0}
  \end{equation}
  
  \item \textbf{La quantité de matières premières :} Elle doit
  être positive.
  \begin{equation} 
  	\forall i, S_i \ge 0 \label{C0}
  \end{equation}
  
  \item \textbf{Le temps d'occupation de chaque machine $i$:} Il doit être
  inférieur au temps de travail
  \begin{equation} 
  	\sum_{j = A}^{F} T_{j,i} . n_j \leq 2.8.60.5 = 4800 \label{C1}
  \end{equation} 
  soit un temps de travail en deux huit, 5 jours par semaine.
  
  \item \textbf{L'utilisation de chaque matière première  $i$:} Elle doit être
  inférieure au stock
  \begin{equation} 
  	\sum_{j = A}^{F} Q_{i,j} . n_j \leq S_i \label{C2}
  \end{equation} 
\end{itemize}

\newpage
\section{Objectif : Comptable}
Le comptable cherche à maximiser les bénefices sous les contraintes définies
précedemment.

\subsection{Modélisation}
Soit $n_i$ le nombre de produit $i$ fabriqué. Le coup fixe de production
n'influant pas sur notre décision, nous ne considérerons que le coût variable de
production. Il est défini par la formule suivante:
\begin{displaymath}
CV(i) = n_i * \left (\sum_{j = 1}^{7} T_{i,j} .
\frac{C_{i,j}}{60} + \sum_{k = 1}^{3} Q_{k,i} . A_{k} \right )
\end{displaymath}
~\\
Le chiffre d'affaire par produit est :
\begin{displaymath}
CA(i) = n_i . V_i
\end{displaymath}
~\\
Par conséquent le bénefice par produit se calcule de la manière suivante :
\begin{eqnarray*}
	B(i) &=& CA(i) - CV(i)\\
	B(i) &=& n_i * \left (V_i - \sum_{j = 1}^{7} T_{i,j} . \frac{C_{i,j}}{60} +
	\sum_{k = 1}^{3} Q_{k,i} . A_{k} \right )
\end{eqnarray*}

\subsection{Décisions}
\newpage
\section{Objectif : Responsable d'atelier}
Le responsable d'atelier cherche à maximiser le nombre d'unités (toutes
catégories confondues) produites sous les contraintes définies précedemment.

\subsection{Modélisation}
Soit $N$ le nombre de produits fabriqués.

\begin{equation}
	N = \sum_{i = A}^{F}
\end{equation} 

\subsection{Décisions}
\newpage
\section{Objectif : Responsable commercial}
Le responsable commercial cherche à équilibrer le nombre
d'unités de ${A, B, C}$ (famille 1) et ${D, E, F}$ (famille 2) afin que ces deux
familles contiennent le même nombre d'unités ( à $\epsilon$
unité(s) près).\\
Autrement dit, l'écart entre le nombre d'unités produite pour la famille A et la famille B doit être inférieur à un seuil $\epsilon$.

\subsection{Modélisation}
Soient :
\begin{itemize}
  \item $N_1$ le nombre de produits de la famille 1 fabriqués.
  \item $N_2$ le nombre de produits de la famille 2 fabriqués.
\end{itemize}

\begin{eqnarray*}
	|N_1 - N_2| &\leq& \epsilon\\
	\Leftrightarrow -\epsilon \leq N_1 - N_2 &\leq& \epsilon\\
	\Leftrightarrow -\epsilon \leq \sum_{i = A}^{C} n_i - \sum_{j = D}^{F} n_j
	&\leq& \epsilon\\
\end{eqnarray*} 
Par concéquent, c'est cette nouvelle contrainte qui, venant s'ajouter aux
contraintes précédentes, va permettre de calculer le nombre d'unités A, B, C,
D, E, et F à fabriquer afin d'équilibrer les deux familles.\\
~\\
Nous obtenons la matrice suivante :
\begin{displaymath}
M = \left(
\begin{array}{cccccc}
1 & 1 & 1 & 1 & 1 & 1\\
\end{array}
\right)
\end{displaymath}
La matrice, très simple, représente la somme des différents produits.\\
La matrice des contraintes devient quant à elle :
INSÉRER MATRICE A MODIFIEE

\subsection{Décisions}

\subsection{Interprétation}
Evidemment, toutes les solutions triviales du type :
\begin{displaymath}
M_p = \left(
\begin{array}{cccccc}
N \pm \epsilon & N \pm \epsilon & N \pm \epsilon & N & N & N\\
\end{array}
\right)
\end{displaymath}
ou encore 
\begin{displaymath}
M_p = \left(
\begin{array}{cccccc}
N & N & N & N \pm \epsilon & N \pm \epsilon & N \pm \epsilon\\
\end{array}
\right)
\end{displaymath}
\begin{center}
\textsl{Où $M_p$ est la matrice du nombre de produit, et $N \in \mathbbm{N}$}
\end{center}
sont des solutions \emph{valables}.\\
~\\
Ceci met en évidence qu'avec les critères définis plus haut, il n'y a pas de
solution plus \og valable\fg ~qu'une autre. Par concéquent nous pouvons :
\begin{itemize}
  \item Choisir une solution au hasard
  \item Augmenter le nombre de critère et notamment ceux en rapport avec les
  stocks disponibles, le prix des matières premières, ou encore le temps
  d'usinage nécessaire.
\end{itemize}

~\newpage
\nTitle{Programmation linéaire multicritère}

\section{Objectifs}
L'objectif est de trouver une solution de compromis entre les différents responsables.
Pour trouver une telle solution nous serons amenés à utiliser la programmation multicritère (\emph{PLM}).
Auparavant, dans la partie 1, nous avons trouvé un optimum pour chaque
responsable indépendam\-ment, ce qui nous conduit à un point de mire. Dans un
monde parfait, ce point de mire respecterait les contraintes de chaque
responsable. Nous devons donc voir si tel est le cas. 

\section{Recherche du point de départ}
Si le point de mire est assez proche de l'ensemble des solutions acceptables,
nous choisirons une solution proche de celle d'un responsable.

Sinon, nous allons calculer la satisfaction de chaque objectif, sachant qu'une
solution a été retenue. Nous devrons alors définir des métriques, correspondant
à cette satisfaction. Par exemple, pour le comptable, cette satisfaction sera
exprimée par le ratio du bénéfice obtenue dans un solution par rapport au
bénéfice maximal.
Ensuite, nous choisirons comme point de départ la solution qui offre le plus de
satisfaction à tout le monde, par exemple en utilisant une moyenne pondérée,
dont la pondération sera basée sur \emph{l'importance} de chaque critère.

\section{Affinement de la solution}
La solution trouvée précédemment peut sûrement être optimisée. Il peut être
intéressant de perdre dans un critère, si cela nous fait gagner beaucoup dans
un autre critère, d'autant plus si ce second critère est jugé plus
\emph{intéressant} que le premier.

\section{Métriques utilisées}
Cette section décrit les métriques utilisées pour caractériser une solution, du
point de vue d'un cadre de l'entreprise. Les solutions pourront ainsi être
comparées entre elles.

\paragraph{Comptable :}
La métrique utilisée sera le pourcentage du bénéfice par rapport au bénéfice
maximum :
$$
M_{Comptable} = \frac{B_{S}}{B_{max}} \times 100
$$

\paragraph{Responsable d'atelier}
La métrique utilisée sera le pourcentage du nombre de produits fabriqués par
rapport au nombre maximum :
$$
M_{Atelier} = \frac{N_{S}}{N_{max}} \times 100
$$

\paragraph{Responsable des stocks}
Pour élaborer la métrique de satisfaction pour le responsable des stocks nous
opterons pour la fonction suivante, qui correspond à ce qui était précédemment
annoncé dans la partie 1 (page \pageref{stocks}).

\begin{figure}[!ht]
\begin{center}
    \includegraphics[width=\linewidth]{multicritere_graphe_stocks.png}
    \caption{Représentation graphique de la métrique pour le responsable des
	stocks.}
	\end{center}
\end{figure}

Cette fonction est décrite par l'expression suivante :
$$
M_{Stocks} = \left\{ 
    \begin{array}{l l l}
	\frac{x}{1192} \text{ si } x \in \text{[0 ; 1192]} \\
	x=1 \text{ si } x \in [1192 ; 1559]\\
	\frac{-x}{1192}+ 1+\frac{1559}{1192} \text{ si } x \in \text{[1559 ;
	    1691]}\\
    \end{array}
\right.
$$

Cette fonction pourrait être justifiée par le fait que plus on s’éloigne des
valeurs admises moins le responsable des stocks est satisfait.


\paragraph{Responsable commercial}
La métrique utilisée sera l'écart par rapport à un équilibre parfait.
Si autant de produit de la famille de produit 1 (comprenant les produits A, B
et C) que de la famille 2 (comprenant les produits D, E et F), la métrique sera
à 100\%.
Si une seule famille de produit est fabriquée, la métrique devra valoir zéro.

Si $F_1$ (respectivement $F_2$ est le nombre de produit de la famille 1
(respectivement famille 2), la métrique sera :
$$
M_{Commercial} = \left( 1 - \frac{|F_1 - F_2|}{F_1 + F_2} \right) \times 100
$$

\section{Utilisation}
Les résultats seront placés dans un tableau de ce type, qui qui permettra d'un
seul coup d'œil de voir la meilleur des solutions.
La colonne en rouge se lira par exemple : 
\begin{center}
«~En suivant la volonté du responsable d'atelier, le comptable aura une
satisfaction de $96.5498\%$~»
\end{center}

\begin{table}[!ht]
    \begin{center}
    \begin{tabular}{|l|c|c|c|c|}
\hline
\cellcolor[gray]{0.9} & Comptable& Atelier & Stock & Commercial  \\
\hline
Comptable			& \cellcolor[gray]{0.9} 100\% & 94.7678\% & 40.0930\%			  & 11.4302\% \\
\hline
Atelier				& \cellcolor{red}   96.5498\% &
\cellcolor[gray]{0.9} 100\% & 32.2825\%			  & 47.9244\% \\
\hline
Stock			        & 74.1546\%		      & 70.8003\%		    & \cellcolor[gray]{0.9} 100\% & 25.2908\% \\
\hline
Commercial			& 81.8654\%		      & 93.4330\%
& 30.6085\%                   & \cellcolor[gray]{0.9} 100\% \\
\hline
    \end{tabular}
    \end{center}
    \caption{Tableau de satisfaction des différents cadres de l'entreprise en
	fonction de la solution retenue.}
\end{table}

On voit bien que le point de mire est éloigné de l'ensemble des solutions
acceptables : aucune solution ne semble satisfaire tous les responsables.

\section{Optimisation}
Nous allons dans cette partie essayer de modifier les solutions précédentes
pour trouver une solution qui tente de maximiser la satisfaction des différents
responsables.

Le but premier d'une entreprise étant de faire du profit, et la meilleur manière de
faire du profit étant souvent maximiser le bénéfice, nous allons donc favoriser
ce critère, en lui appliquant un léger coefficient.

Nous allons par la même regarder l'évolution des autres critères lorsque ce
coefficient évolue, c'est à dire recalculer les différentes métriques.

Il apparaît plausible qu'avoir un \emph{stock non optimal} peut être acceptable
si c'est pour avoir un bénéfice plus important, dans une certaine mesure. Une
dégradation de la satisfaction du responsable des stock sera donc moins
pénalisant, en terme de qualité de solution, qu'une baisse dans la satisfaction
du commercial (produire sans vendre ne sert pas à grand chose, en terme de
rentabilité).

De la même manière, produire un nombre de produit optimal peut être
intéressant, mais cela ne doit pas aller à l'encontre du profit et de la
mauvaise répartition de la production.

Nous pouvons donc donner des coefficients aux critères :

\begin{table}[h!]
\begin{center}
\begin{tabular}{|l||c|c|c|c|}
\hline
    Responsable & Comptable & Atelier & Commercial & Stocks \\
	\hline
    Coefficient & 1.2	    & 0.8     & 1.2	& 0.8 \\
	\hline
	\end{tabular}
	\end{center}
\caption{Coefficient associés aux critères des différents responsables}
\end{table}

\subsection{Méthodologie d'optimisation}
Le procédé sera itératif, et tentera de contenter au mieux le comptable et le
commercial (comme expliqué ci-dessus). Il s'agira donc de partir de la solution
du comptable (celle qui maximise le bénéfice), en essayant d'améliorer la
métrique du commercial, tout en restant dans le domaines des solutions possibles
au vu des contraintes. Il faudra toutefois tâcher de ne pas trop dégrader les
autres critères.
\subsection{Résultats}
Présenter un tableau.


\nTitle{Analyse multicritère}

Des deux parties précédentes, émergent 8 propositions de gestion de l'atelier.
Le but de cette partie sera de sélectionner la meilleure solution en fonction de 4 critères.

\section{Méthode choisie}

La méthode de résolution choisie sera Électre III car elle englobe les 2 précédentes.
On pourras ainsi fournir au client la méthode sélectionné comme la plus optimale ainsi qu'une ou plusieurs méthodes alternatives.

Pour réduire au maximum l'échéance, nous avons parallélisé au maximum les flux de travail.
Nous avons donc dès le début du projet commencé par coder sous Matlab un algorithme de résolution de Electre 3.

\section{Mise en œuvre de la méthode}
L'algorithme suivant implémente la méthode Électre \rmnum{3}. À partir de la matrice des jugements, il remet à l'echelle les notes des critères en fonction des poids.

\addCode{../SourcesMatlab/electreSnippet1.m}{matlab}

Puis il calcule les matrices de concordance et de discordance.

\addCode{../SourcesMatlab/electreSnippet2.m}{matlab}


Pour finir, il établit la matrice des surclassement en fonction de ces deux dernières matrices.

\addCode{../SourcesMatlab/electreSnippet3.m}{matlab}
o
Un graphe est ensuite généré à partir de cette matrice.
On obtient ainsi rapidement la meilleurs solution en tête de graphe, et grâce à la méthode du classement et du classement inverse, on peut obtenir l'ordre des solutions.

\section{Solution proposée}

Dans un premier temps, sans prendre en compte l'étude des 2 première parties, la meilleur solution, serait la solution A.

\begin{figure}[!ht]
\includegraphics{../SourcesMatlab/electre3-1.pdf}
\caption{Graphe des surclassement sans prise en compte des poids de chaque critère}
\end{figure}

Dans un second temps, en prenant en compte les poids apporté par l'étude des 2 première parties.


\newpage
\nTitle{Annexe}
\section{Code source}
\vspace*{\fill}
	\lstinputlisting{../SourcesMatlab/Comptable.m}
\vspace*{\fill}

\newpage
\vspace*{\fill}
	\lstinputlisting{../SourcesMatlab/Atelier.m}
\vspace*{\fill}


\graphicspath{{../SourcesMatlab/}}
\graphicspath{{.}}

\end{document}
