\nTitle{Programmation Linéaire monocritère}

\section{Données}
Soient :
\begin{itemize}
  \item \textbf{T} la matrice des temps unitaires d'usinage d'un produit sur une
  machine (minutes) (\textsl{C.f. Table 1}).
  \item \textbf{Q} la matrice de quantité de matières premières par produit
  (\textsl{C.f. Table 2}).
  \item \textbf{S} la matrice des quantité maximum de matières premières
  (\textsl{C.f. Table 3}).
  \item \textbf{V} la matrice des prix de vente des produits finis (\textsl{C.f.
  Table 4})
  \item \textbf{A} la matrice des prix d'achat des matières premières.
  \item \textbf{C} la matrice des coûts horaires des machines (\textsl{C.f.
  Table 5}).
\end{itemize}
 
\subsection{Contraintes}
Considérons :
\begin{itemize}
  \item 6 produits identifiés chacun par une lettre $i \in {A, B, C, D, E, F}$
  \item 7 machines identifiée chacune par un nombre $j \in {1, 2, 3, 4, 5, 6 ,7}$
  \item 3 matières premières identifiée chacune par un nombre $k \in {1, 2, 3}$
  \item $n$, vecteur colonne du nombre d'unités fabriquées pour chaque produit
\end{itemize}
~\\
L'ensemble de la chaine de production est régie par les contraintes suivantes
:\\
\begin{itemize}
  \item \textbf{Le nombre de produits usinés de type $i$ :} Il doit être non nul
  \begin{equation} 
  	n_i \ge 0, \forall i \in {A, B, C, D, E, F} \label{C0}
  \end{equation}
  
  \item \textbf{Le temps d'occupation de chaque machine $i$:} Il doit être
  inférieur au temps de travail
  \begin{equation} 
  	\sum_{j = A}^{F} T_{j,i} \times n_j \leq 2 \times 8 \times 60 \times 5 = 4800, \forall i \in {1, 2, 3, 4, 5, 6 ,7} \label{C1}
  \end{equation} 
  soit un temps de travail en deux huit, 5 jours par semaine.
  
  \item \textbf{L'utilisation de chaque matière première  $i$:} Elle doit être
  inférieure au stock
  \begin{equation} 
  	\sum_{j = A}^{F} Q_{i,j} \times n_j \leq S_i, \forall i \in {1,2,3} \label{C2}
  \end{equation} 
\end{itemize}

\newpage
\subsubsection{Modélisation sous forme matricielle}
Pour donner au problème une forme standard, nous allons le modéliser par des inéquations et des produits matriciels.
Les contraintes C0, C1 et C2 se traduisent trivialement de la manière suivante :

\begin{equation}
A.n \leq b
\end{equation}
Avec :
\[
A =
\begin{pmatrix}
\\
\\
\quad & \quad & -I & \quad & \quad \\
\\
\\
\\
\hline
\\
\\
\\
\quad & \quad & T^{t} & \quad & \quad \\
\\
\\
\\
\hline
\\
\quad & \quad & Q & \quad & \quad \\
\\
\end{pmatrix}
, b = 
\begin{pmatrix}
0 \\ 0 \\ 0 \\ 0 \\ 0 \\ 0 \\
\hline
4800 \\ 4800 \\ 4800 \\ 4800 \\ 4800 \\ 4800 \\ 4800 \\
\hline
\\
S^{t} \\
\\
\end{pmatrix}
\]

Ce qui nous donne plus concrètement les matrices suivantes :
\[
A =
\begin{pmatrix}
   -1 &    0 &    0 &    0 &    0 &    0 \\
    0 &   -1 &    0 &    0 &    0 &    0 \\
    0 &    0 &   -1 &    0 &    0 &    0 \\
    0 &    0 &    0 &   -1 &    0 &    0 \\
    0 &    0 &    0 &    0 &   -1 &    0 \\
    0 &    0 &    0 &    0 &    0 &   -1 \\
\hline
    8 &   15 &    0 &    5 &    0 &   10 \\
    0 &    1 &    2 &   15 &    7 &   12 \\
    8 &    1 &   11 &    0 &   10 &   25 \\
    2 &   10 &    5 &    4 &   13 &    7 \\
    5 &    0 &    0 &    0 &   10 &   25 \\
    5 &    5 &    3 &   12 &    8 &    0 \\
    5 &    3 &    5 &    8 &    0 &    0 \\
\hline
    1 &    2 &    1 &    5 &    0 &    2 \\
    2 &    2 &    1 &    0 &    2 &    1 \\
    1 &    0 &    3 &    2 &    2 &    0 \\
\end{pmatrix}
, b = 
\begin{pmatrix}
0 \\ 0 \\ 0 \\ 0 \\ 0 \\ 0 \\
\hline
4800 \\ 4800 \\ 4800 \\ 4800 \\ 4800 \\ 4800 \\ 4800 \\
\hline
350 \\ 620 \\ 485
\end{pmatrix}
\]

\newpage
\section{Objectif : Comptable}
Le comptable cherche à maximiser les bénefices sous les contraintes définies
précedemment.

\subsection{Modélisation}
Soit $n_i$ le nombre de produit $i$ fabriqué. Le coup fixe de production
n'influant pas sur notre décision, nous ne considérerons que le coût variable de
production. Il est défini par la formule suivante:
\begin{displaymath}
CV(i) = n_i * \left (\sum_{j = 1}^{7} T_{i,j} .
\frac{C_{i,j}}{60} + \sum_{k = 1}^{3} Q_{k,i} . A_{k} \right )
\end{displaymath}
~\\
Le chiffre d'affaire par produit est :
\begin{displaymath}
CA(i) = n_i . V_i
\end{displaymath}
~\\
Par conséquent le bénefice par produit se calcule de la manière suivante :
\begin{eqnarray*}
	B(i) &=& CA(i) - CV(i)\\
	B(i) &=& n_i * \left (V_i - \sum_{j = 1}^{7} T_{i,j} . \frac{C_{i,j}}{60} +
	\sum_{k = 1}^{3} Q_{k,i} . A_{k} \right )
\end{eqnarray*}

\subsection{Décisions}
\newpage
\section{Objectif : Responsable d'atelier}
Le responsable d'atelier cherche à maximiser le nombre d'unités (toutes
catégories confondues) produites sous les contraintes définies précedemment.

\subsection{Modélisation}
Soit $N$ le nombre de produits fabriqués.

\begin{equation}
	N = \sum_{i = A}^{F}
\end{equation} 

\subsection{Décisions}
\newpage
\section{Objectif : Responsable des stocks}
Le responsable des stocks cherche à minimiser le nombre de de produits dans
son stock sous les contraintes définies précedemment.

\subsection{Modélisation}
Soit $Stock(n_{i})$ le nombre d'unités de stock nécessaires pour stocker les
produits fabriqués et la matière première nécessaire. Cette fonction est la
somme des produits fabriqués à laquelle on ajoute la quantité de matières
premières nécessaire à la fabrication.

On suppose qu'un produit frabriqué correspond à une unité de stock.

On a ainsi la formule suivante, où $n_{i}$ est la quantité de produit usiné
(pour chaque produit $i$), et $Q_{j,i}$ est la quantité de matière première
par produit pour chaque produit $i$ et chaque matière première $j$.

\begin{equation}
	Stock(n) = \sum_{i} (n_{i} + n_{i} \times \sum_{j} Q_{j,i})
\end{equation}

La représentation matricielle de cette fonction sera alors :

\begin{equation}
	M_S = \begin{pmatrix}
		1 & 1 & 1 & 1 & 1 & 1
	\end{pmatrix} + (
	\begin{pmatrix}
		1 & 1 & 1
	\end{pmatrix}
	\times Q)
\end{equation}

Le résultat trivial est :
\begin{equation}
    \begin{pmatrix}
	0 \\ 0 \\ 0 \\ 0 \\ 0 \\ 0
    \end{pmatrix}
\end{equation}

En effet, en l'absence de production, aucun stock n'est nécessaire. Ce résultat
n'est pas satisfaisant. Nous allons donc nous intéresser à un second critère :
les bénéfices de l'entreprise. Pour ce faire, nous allons nous intéresser aux
valeurs obtenues en imposant un minimum de bénéfices. En réalisant les calculs
sur plusieurs échelons, on obtient un résultat linéaire par morceaux. Le
raisonement adopté est similaire à celui développé dans la section
\ref{sec:monocrit_resp_com}.

\begin{figure}[h!]
	\includegraphics[width=\textwidth]{SourcesMatlab/graphe_resp_stocks.png}
	\caption{Graphe de l'évolution des stocks en fonction du bénéfice}
\end{figure}


\subsection{Décisions}
De nombreuses valeurs sont équivalentes et ne peuvent être départagées selon des
critères mathématiques. Les choix possibles se situent dans le deuxième morceau
de courbe : en dessous, les bénéfices sont trop bas, au dessus, les besoins de
stockage augmentent beaucoup plus que les bénéfices.

On a donc une valeur comprise entre 50\% et 98\% de bénéfices. Pour 75\% du
bénéfice maximum, on obtient le nombre de produits suivants :

\begin{equation}
\begin{pmatrix}
1,91903382074088 \times 10^{-10} \\
2,63753463514149 \times 10^{-10} \\ 
1,89174897968769 \times 10^{-10} \\
1,23691279441118 \times 10^{-10} \\ 
124,634235411818 \\
142,146305832495 
\end{pmatrix}
\end{equation}
pour une quantité d'unités en stock de 1191,75640039347.




\input{ResponsableCommercial.utf8.tex}
