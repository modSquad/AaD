\nTitle{Programmation Linéaire monocritère}

\section{Données}
Soient :
\begin{itemize}
  \item \textbf{T} la matrice des temps unitaires d'usinage d'un produit sur une
  machine (minutes) (\textsl{C.f. Table 1}).
  \item \textbf{Q} la matrice de quantité de matières premières par produit
  (\textsl{C.f. Table 2}).
  \item \textbf{S} la matrice des quantité maximum de matières premières
  (\textsl{C.f. Table 3}).
  \item \textbf{V} la matrice des prix de vente des produits finis (\textsl{C.f.
  Table 4})
  \item \textbf{A} la matrice des prix d'achat des matières premières.
  \item \textbf{C} la matrice des coûts horaires des machines (\textsl{C.f.
  Table 5}).
\end{itemize}

\subsection{Contraintes}
L'ensemble de la chaine de production est régie par les contraintes suivantes :
\begin{itemize}
  \item \textbf{Le temps d'occupation de chaque machine $i$:} Il doit être
  inférieur au temps de travail
  $$
  	\sum_{j = A}^{F} T_{j,i} . n_j \leq 2.8.60.5 = 4800 (C_1)
  $$ soit un temps de travail en deux huit, 5 jours par semaine.
  \item \textbf{L'utilisation de chaque matière première  $i$:} Elle doit être
  inférieure au stock
  $$
  	\sum_{j = A}^{F} Q_{i,j} . n_j \leq S_i (C_2)
  $$
\end{itemize}

\section{Objectif : Comptable}
Le comptable cherche à maximiser les bénefices sous les contraintes définies
précedemment.

\subsection{Modélisation}
Soit $n_i$ le nombre de produit $i$ fabriqué. Le coup fixe de production
n'influant pas sur notre décision, nous ne considérerons que le coût variable de
production. Il est défini par la formule suivante: $$
CV(i) = n_i * \left (\sum_{j = 1}^{7} T_{i,j} .
\frac{C_{i,j}}{60} + \sum_{k = 1}^{3} Q_{k,i} . A_{k} \right )
$$
~\\
Le chiffre d'affaire par produit est :
$$
CA(i) = n_i . V_i
$$
~\\
Par conséquent le bénefice par produit se calcule de la manière suivante :
$$
B(i) = CA(i) - CV(i)
B(i) = n_i * \left (V_i - \sum_{j = 1}^{7} T_{i,j} .
\frac{C_{i,j}}{60} + \sum_{k = 1}^{3} Q_{k,i} . A_{k} \right )
$$