\nTitle{Programmation linéaire multicritère}

L'objectif est de trouver une solution de compromis entre les différents responsables.
Pour trouver une telle solution nous serons amenés à utiliser la programmation multicritère (\emph{PLM}).
Auparavant dans la partie 1, nous avons trouvé un optimum pour chaque
responsable indépendam\-ment, ce qui nous conduit à un point de mire. Dans un
monde parfait ce point de mire respecterait les contraintes de chaque
responsable. Nous devons donc voir si telle est le cas. 


\section{Recherche du point de départ}
Si le point de mire est assez proche de l'ensemble des solutions acceptables,
nous choisirons une solution proche de celle d'un responsable.

Sinon, nous allons calculer la satisfaction de chaque objectif, sachant qu'une
solution a été retenue. Nous devrons alors définir des métriques, correspondant
à cette satisfaction. Par exemple, pour le comptable, cette satisfaction sera
exprimée par le ratio du bénéfice obtenue dans un solution par rapport au
bénéfice maximal.
Ensuite, nous choisirons comme point de départ la solution qui offre le plus de
satisfaction à tout le monde, par exemple en utilisant une moyenne pondérée,
dont la pondération sera basée sur \emph{l'importance} de chaque critère.

\section{Affinement de la solution}
La solution trouvée précédemment peut sûrement être optimisée. Il peut être
intéressant de perdre dans un critère, si cela nous fait gagner beaucoup dans
un autre critère, d'autant plus si ce second critère est jugé plus
\emph{intéressant} que le premier.

\section{Métriques utilisées}
Cette section décrit les métriques utilisées pour caractériser une solution, du
point de vue d'un cadre de l'entreprise. Les solutions pourront ainsi être
comparées entre elles.

\paragraph{Comptable :}
La métrique utilisée sera le pourcentage du bénéfice par rapport au bénéfice
maximum :
$$
M_{Comptable} = \frac{B_{S}}{B_{max}} \times 100
$$

\paragraph{Responsable d'atelier}
La métrique utilisée sera le pourcentage du nombre de produits fabriqués par
rapport au nombre maximum :
$$
M_{Atelier} = \frac{N_{S}}{N_{max}} \times 100
$$

\paragraph{Responsable des stocks}
La métrique utilisée sera la somme de la valeur du stock (le prix unitaire
de chaque matière première avec la quantité de cette matière
première en stock), plus le nombre de produit fabriqués (et donc en stock),
multiplié par la quantité nécessaire de matière première pour fabriquer chaque
produit. On peut imaginer que plus un produit nécessite de matière première, et
plus cette matière première est chère, plus il sera cher à stocker.

Il pourra être nécessaire de mettre cette métrique sous forme de pourcentage,
pour en faciliter la lecture.

$$
\begin{cases}
Q_MPi : \text{quantité de matière première } i \text{ dans le stock à un instant} t.\\
Pu_{Mpi} : \text{prix de la matière première } i.\\
QNec_{i_{MPj}} : \text{quantité nécessaire de matière première $j$ pour le produit $i$.}
\end{cases}
$$

$$
M_{Stock} =  \left( 
	     \sum_{i=1}^{3} Q_{MPi} \times Pu_{Mpi} + 
	     \sum_{i=A}^{F} \sum_{j=1}^{3} QNec_{i_{MP1j}} \times Pu_{Mpj}
	     \right)^{-1}
$$

$$
M_{Stock} = \left(
	    1 - \frac{Q_{produit} + Q_{MP}}{Q_{MP_{Max}}} 
	    \right) \times 100
$$
\paragraph{Responsable commercial}
La métrique utilisée sera l'écart par rapport à un équilibre parfait.
Si autant de produit de la famille de produit 1 (comprenant les produits A, B
et C) que de la famille 2 (comprenant les produits D, E et F), la métrique sera
à 100\%.
Si une seule famille de produit est fabriquée, la métrique devra valoir zéro.

Si $F_1$ (respectivement $F_2$ est le nombre de produit de la famille 1
(respectivement famille 2), la métrique sera :
$$
M_{Commercial} = \left( 1 - \frac{|F_1 - F_2|}{F_1 + F_2} \right) \times 100
$$

\section{Utilisation}
Les résultats seront placés dans un tableau de ce type, qui qui permettra d'un
seul coup d'œil de voir la meilleur des solutions.
La colonne en rouge se lira par exemple : 
\begin{center}
«~En suivant la volonté du responsable d'atelier, le comptable aura une
satisfaction de $n\%$~»
\end{center}

Un calcul de moyenne pourra être un premier indicateur de qualité d'une
solution. Cette moyenne pourra être pondérée dans le future.

\begin{table}[h!]
    \begin{center}
    \begin{tabular}{|l|c|c|c|c|c|}
	\hline
	\cellcolor[gray]{0.9} & Comptable& Atelier & Stock & Commercial &
	Moyenne\\
	\hline
	Comptable & \cellcolor[gray]{0.9} 100\% & 2 & 3 & 4 & \\
	\hline
	Atelier & \cellcolor{red}1 & \cellcolor[gray]{0.9} 100\%  & 3 & 4 & \\
	\hline
	Stock & 1 & 2 & \cellcolor[gray]{0.9} 100\%  & 4 & \\
	\hline 
	Commercial & 1 & 2 & 3 & \cellcolor[gray]{0.9} 100\% & \\
	\hline
    \end{tabular}
    \end{center}
    \caption{Tableau de satisfaction des différents cadres de l'entreprise en
	fonction de la solution retenue.}
\end{table}



\begin{tabular}{|l||c|c|c|c||c|c|c|c||c|c|c|c||c|c|c|c|}
\hline
~ & \multicolumn{4}{|c||}{Comptable}
  & \multicolumn{4}{|c||}{Atelier}
  & \multicolumn{4}{|c||}{Stock}
  & \multicolumn{4}{|c|}{Commercial} \\
  \hline
Comptable & 1 & 2 & 3 & 4
	  & 1 & 2 & 3 & 4 
	  & 1 & 2 & 3 & 4
	  & 1 & 2 & 3 & 4 \\
\hline
Atelier & 1 & 2 & 3 & 4
	  & 1 & 2 & 3 & 4 
	  & 1 & 2 & 3 & 4
	  & 1 & 2 & 3 & 4 \\
\hline
Stock & 1 & 2 & 3 & 4
      & 1 & 2 & 3 & 4 
      & 1 & 2 & 3 & 4
      & 1 & 2 & 3 & 4 \\
\hline 
Commercial & 1 & 2 & 3 & 4
	   & 1 & 2 & 3 & 4 
	   & 1 & 2 & 3 & 4
	   & 1 & 2 & 3 & 4 \\
\hline


\end{tabular}

